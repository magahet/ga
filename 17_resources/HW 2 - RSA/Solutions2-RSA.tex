\documentclass{article}
\usepackage[T1]{fontenc}
\usepackage{amssymb, amsmath, graphicx, subfigure, enumerate}
\usepackage{amsthm,alltt} 
\usepackage[margin=1.25in]{geometry} %geometry (sets margin) and other useful packages
\usepackage{graphicx,ctable,booktabs}
\usepackage{mathtools}
\usepackage[boxed]{algorithm2e}
\usepackage{mathdots}
\usepackage{fancyhdr} %Fancy-header package to modify header/page numbering
\usepackage{cleveref}

\setlength{\oddsidemargin}{.25in}
\setlength{\evensidemargin}{.25in}
\setlength{\textwidth}{6in}
\setlength{\topmargin}{-0.4in}
\setlength{\textheight}{8.5in}



\newcommand{\heading}[6]{
  \renewcommand{\thepage}{\arabic{page}} % used to be #1-\arabic{page}
  \noindent
  \begin{center}
  \framebox{
    \vbox{
      \hbox to 5.78in { \textbf{#2} \hfill #3 }
      \vspace{4mm}
      \hbox to 5.78in { {\Large \hfill #6  \hfill} }
      \vspace{2mm}
      \hbox to 5.78in { \textit{Instructor: #4 \hfill #5} }
    }
  }
  \end{center}
  \vspace*{4mm}
}

%Redefining sections as problems
\makeatletter
\newenvironment{problem}{\@startsection
       {section}
       {2}
       {-.2em}
       {-3.5ex plus -1ex minus -.2ex}
       {2.3ex plus .2ex}
       {\pagebreak[3]%forces pagebreak when space is small; use \eject for better results
       \large\bf\noindent{Problem }
       }
       }
       %{%\vspace{1ex}\begin{center} \rule{0.3\linewidth}{.3pt}\end{center}}
       %\begin{center}\large\bf \ldots\ldots\ldots\end{center}}
\makeatother


\newtheorem{theorem}{Theorem}[section]
\newtheorem{definition}[theorem]{Definition}
\newtheorem{remark}[theorem]{Remark}
\newtheorem{lemma}[theorem]{Lemma}
\newtheorem{corollary}[theorem]{Corollary}
\newtheorem{proposition}[theorem]{Proposition}
\newtheorem{claim}[theorem]{Claim}
\newtheorem{observation}[theorem]{Observation}
\newtheorem{fact}[theorem]{Fact}
\newtheorem{assumption}[theorem]{Assumption}

%\newenvironment{proof}{\noindent{\bf Proof:} \hspace*{1mm}}{
% \hspace*{\fill} $\Box$ }
%\newenvironment{proof_of}[1]{\noindent {\bf Proof of #1:}
% \hspace*{1mm}}{\hspace*{\fill} $\Box$ }
%\newenvironment{proof_claim}{\begin{quotation} \noindent}{
% \hspace*{\fill} $\diamond$ \end{quotation}}

\newcommand{\problemset}[3]{\heading{#1}{\classname}{#2}{\instructor}{#3}{Problem Set #1}} % Don't change this line
%%%%%%%%%%%%%%%%%%%%%%%%%% Change this stuff below, don't change the line above this one
\newcommand{\problemsetnum}{2}            % problem set number
\newcommand{\duedate}{09/11/2017} % problem set deadline
\newcommand{\studentname}{Tiancheng Gong}      % name of student (i.e., you)
\newcommand{\classname}{CS 8803 GA}
\newcommand{\instructor}{Prof. Eric Vigoda}
%%%%%%%%%%%%%%%%%%%%%%%%%%

\pagestyle{fancy}
%\addtolength{\headwidth}{\marginparsep} %these change header-rule width
%\addtolength{\headwidth}{\marginparwidth}
\lhead{\classname} %Problem \thesection}
\chead{} 
\rhead{\thepage} 
\lfoot{\small\scshape \classname}
\cfoot{} 
\rfoot{\footnotesize PS \#\problemsetnum} 
\renewcommand{\headrulewidth}{.3pt} 
\renewcommand{\footrulewidth}{.3pt}
\setlength\voffset{-0.25in}
\setlength\textheight{648pt}


\newcommand{\sit}{\shortintertext}
\newcommand\deq{\mathrel{\overset{\makebox[0pt]{\mbox{\normalfont\tiny\sffamily def}}}{=}}}
\newcommand{\ones}{\mathbbm{1}}
\newcommand{\e}{\vec{e}}
\newcommand{\tr}{\text{tr}}
\newcommand{\bs}{\boldsymbol}
\mathchardef\mhyphen="2D
\newcommand{\C}{\mathbb{C}}
\newcommand{\R}{\mathbb{R}}
\newcommand{\II}{\mathcal{I}}
\newcommand{\FF}{\mathcal{F}}
\newcommand{\X}{\mathcal{X}}
\newcommand{\Y}{\mathcal{Y}}
\newcommand{\ra}{\rightarrow}
\newcommand{\Ra}{\Rightarrow}
\newcommand{\PP}{\mathbb{P}}
\newcommand{\sse}{\subseteq}
\newcommand{\eps}{\epsilon}
\newcommand{\N}{\mathcal{N}}
\newcommand{\poly}{\textup{poly}}

\newcommand{\dom}{\textup{dom}}

\renewcommand{\thesubsection}{\thesection.\roman{subsection}}


% auto sized delimiters
\newcommand{\Br}[1]{\left\{#1\right\}}
\newcommand{\br}[1]{\left[#1\right]}
\newcommand{\pr}[1]{\left(#1\right)}
\newcommand{\ceil}[1]{\left\lceil#1\right\rceil}
\newcommand{\floor}[1]{\left\lfloor#1\right\rfloor}
\newcommand{\abs}[1]{\left|#1\right|}
\newcommand{\sgn}{\textup{sgn}}

%default delimiter for Pr and E
\DeclarePairedDelimiter{\defaultDelim}{[}{]}

\DeclareMathOperator{\capPr}{Pr}
\renewcommand{\Pr}[2][]{\capPr_{#1}\defaultDelim*{#2}}
\DeclareMathOperator{\capE}{E}
\newcommand{\E}[2][]{\capE_{#1}\defaultDelim*{#2}}
\DeclareMathOperator{\capVar}{Var}
\newcommand{\Var}[2][]{\capVar_{#1}\defaultDelim*{#2}}

%\DeclareMathOperator*{}{} puts subscript below


%%%%%%%%%%%%%%%%%%%%%%%%%%%%%%%%%%%%%%%%%%%%%%%%%
\begin{document}
\problemset{\problemsetnum}{\duedate}{\studentname}

\begin{problem} {DPV 1.12}
\textbf{Answer:}\\\\
Since 3 is prime, by Fermat's Little Theorem (or by observation) we know that
$2^2\equiv 1 \mod 3$.  We'll use this fact in the following manner: \\
	\textbf{Answer:}\\\\
	\begin{equation} \nonumber
	2^{2^{2006}} \equiv 2^{2\times 2^{2005}} \equiv \left(2^2\right)^{2^{2005}} \equiv (1)^{2^{2005}} \equiv 1 \mod 3
\end{equation}
\end{problem}

\begin{problem} {DPV 1.25}
By Fermat's Little Theorem, since 127 is prime we know that $2^{126}\equiv 1 \mod 127$.  To use this fact we need to find the inverse of $2 \mod 127$.  Since $gcd(2,127)=1$ hence $2^{-1}\mod 127$ exists, and
observe that $2^{-1} \equiv 64 \mod 127$.  Using these facts we have: \\
	\textbf{Answer:}\\\\
	\begin{equation} \nonumber
	2^{125} \equiv 2^{126}\times 2^{-1} \equiv 1\times 64 \equiv 64  \mod 127
	\end{equation}
\end{problem}

\begin{problem} {DPV 1.20}
	\textbf{Answer:}
	\begin{enumerate}
		\item
		$4 \times 20 - 1 \times 79 = 1 \quad \Rightarrow \quad 20^{-1} \equiv 4 \mod 79$
		\item
		$21 \times 3 - 1 \times 62 = 1 \quad \Rightarrow \quad 3^{-1} \equiv 21 \mod 62$
		\item
		The inverse doesn't exist since $\gcd(21, 91) = 7 \neq 1$.
		\item
		$5 \times 14 - 3 \times 23 = 1 \quad \Rightarrow \quad 5^{-1} \equiv 14 \mod 23$
	\end{enumerate}
\end{problem}

\begin{problem} {DPV 1.22}
	\textbf{Claim: If $a$ has an inverse modulo $b$, then $b$ has an inverse modulo $a$.}\\\\
	\textbf{Proof:}\\\\
	If $a$ has an inverse modulo $b$, then $\gcd(a, b) = 1$, which indicates $b$ has an inverse modulo $a$ as well.\\
	Alternative proof: \\
	Let $c \in \mathbb{Z}$ such that $c \equiv a^{-1} \mod b$, then there exists $d \in \mathbb{Z}$ such that $ca + db = 1$.\\
	Mod both sides with $a$ and we get $ca + db \equiv db \equiv 1 \mod a$, which implies $d$ is an inverse of $b$ modulo $a$. The claim is proved.
\end{problem}

\begin{problem} {DPV 1.28}
	\textbf{Answer:}\\\\
	Given $p = 7$ and $q = 11$, we have $(p-1)(q-1) = 60$.\\
	We try $e = 2, 3, 5, 7, \dots$, and $e=7$ is the first one that has an inverse module 60 since $\gcd(7, 60) = 1$. And using the Extended-Euclid Algorithm covered in the lecture, we can find $d \in \mathbb{Z}$ such that $d \equiv e^{-1} \mod 60$. The answer is $d = 43 \equiv 7^{-1} \mod 60$, which can be verified by $43 \times 7 - 5 \times 60 = 1$.
\end{problem}

\begin{problem} {DPV 1.42}
	\textbf{Claim: The new cryptosystem using only $p$ is not secure.}\\\\
	\textbf{Proof:}\\\\
	Let $p$ be an $n$-bit number.\\\\
	Given $p, e$ and $m^e \mod p$, we can first compute the inverse of $e$ modulo $p-1$ using the Extended-Euclid Algorithm in $O(n^3)$ time. Let the inverse be $a$, and we have $ae + b (p-1) = 1$. \\\\
	According to the Fermat's Little Theorem, we have $m^{p-1} \equiv 1 \mod p$ since $p$ is a prime.\\\\
	Thus we can decrypt the message by $(m^e)^a \equiv m^{ae} \equiv m \cdot m^{-b(p-1)} \equiv m \mod p$. Since we are given $m^e \mod p$, which is at most $\log(p)$ bits and $a \leq p-1$ (otherwise we can subtract $p-1$ from $a$ to get a smaller inverse), the computation time for this step is $O(n^3)$.\\\\
	The total running time for this algorithm is $O(n^3) + O(n^3) = O(n^3)$, which is a polynomial time for the input space ($\log p = n$).
\end{problem}

\end{document}