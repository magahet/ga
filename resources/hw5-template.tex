\documentclass{article}
\usepackage[T1]{fontenc}
\usepackage{amssymb, amsmath, graphicx, subfigure, enumerate}
\usepackage{amsthm,alltt} 
\usepackage[margin=1.25in]{geometry} %geometry (sets margin) and other useful packages
\usepackage{graphicx,ctable,booktabs}
\usepackage{mathtools}
\usepackage[boxed]{algorithm2e}
\usepackage{mathdots}
\usepackage{fancyhdr} %Fancy-header package to modify header/page numbering
\usepackage{cleveref}

\setlength{\oddsidemargin}{.25in}
\setlength{\evensidemargin}{.25in}
\setlength{\textwidth}{6in}
\setlength{\topmargin}{-0.4in}
\setlength{\textheight}{8.5in}



\newcommand{\heading}[6]{
  \renewcommand{\thepage}{\arabic{page}} % used to be #1-\arabic{page}
  \noindent
  \begin{center}
  \framebox{
    \vbox{
      \hbox to 5.78in { \textbf{#2} \hfill #3 }
      \vspace{4mm}
      \hbox to 5.78in { {\Large \hfill #6  \hfill} }
      \vspace{2mm}
      \hbox to 5.78in { \textit{Instructor: #4 \hfill #5} }
    }
  }
  \end{center}
  \vspace*{4mm}
}

%Redefining sections as problems
\makeatletter
\newenvironment{problem}{\@startsection
       {section}
       {2}
       {-.2em}
       {-3.5ex plus -1ex minus -.2ex}
       {2.3ex plus .2ex}
       {\pagebreak[3]%forces pagebreak when space is small; use \eject for better results
       \large\bf\noindent{Problem }
       }
       }
       %{%\vspace{1ex}\begin{center} \rule{0.3\linewidth}{.3pt}\end{center}}
       %\begin{center}\large\bf \ldots\ldots\ldots\end{center}}
\makeatother


\newtheorem{theorem}{Theorem}[section]
\newtheorem{definition}[theorem]{Definition}
\newtheorem{remark}[theorem]{Remark}
\newtheorem{lemma}[theorem]{Lemma}
\newtheorem{corollary}[theorem]{Corollary}
\newtheorem{proposition}[theorem]{Proposition}
\newtheorem{claim}[theorem]{Claim}
\newtheorem{observation}[theorem]{Observation}
\newtheorem{fact}[theorem]{Fact}
\newtheorem{assumption}[theorem]{Assumption}

%\newenvironment{proof}{\noindent{\bf Proof:} \hspace*{1mm}}{
% \hspace*{\fill} $\Box$ }
%\newenvironment{proof_of}[1]{\noindent {\bf Proof of #1:}
% \hspace*{1mm}}{\hspace*{\fill} $\Box$ }
%\newenvironment{proof_claim}{\begin{quotation} \noindent}{
% \hspace*{\fill} $\diamond$ \end{quotation}}

\newcommand{\problemset}[3]{\heading{#1}{\classname}{#2}{\instructor}{#3}{Problem Set #1}} % Don't change this line
%%%%%%%%%%%%%%%%%%%%%%%%%% Change this stuff below, don't change the line above this one
\newcommand{\problemsetnum}{5}            % problem set number
\newcommand{\duedate}{2/26/2018} % problem set deadline
\newcommand{\studentname}{Student Name: }      % name of student (i.e., you)
\newcommand{\classname}{CS 8803 GA -- HW 5.  \ \ Due: \duedate \ \ \ Name:   }
%%%%%%%%%%%%%%%%%%%%%%%%%%

\pagestyle{fancy}
%\addtolength{\headwidth}{\marginparsep} %these change header-rule width
%\addtolength{\headwidth}{\marginparwidth}
\lhead{\classname} %Problem \thesection}
\chead{} 
\rhead{\thepage} 
%\lfoot{\small\scshape \classname}
\cfoot{} 
%\rfoot{\footnotesize PS \#\problemsetnum} 
\renewcommand{\headrulewidth}{.3pt} 
\renewcommand{\footrulewidth}{.3pt}
\setlength\voffset{-0.25in}
\setlength\textheight{648pt}


\newcommand{\sit}{\shortintertext}
\newcommand\deq{\mathrel{\overset{\makebox[0pt]{\mbox{\normalfont\tiny\sffamily def}}}{=}}}
\newcommand{\ones}{\mathbbm{1}}
\newcommand{\e}{\vec{e}}
\newcommand{\tr}{\text{tr}}
\newcommand{\bs}{\boldsymbol}
\mathchardef\mhyphen="2D
\newcommand{\C}{\mathbb{C}}
\newcommand{\R}{\mathbb{R}}
\newcommand{\II}{\mathcal{I}}
\newcommand{\FF}{\mathcal{F}}
\newcommand{\X}{\mathcal{X}}
\newcommand{\Y}{\mathcal{Y}}
\newcommand{\ra}{\rightarrow}
\newcommand{\Ra}{\Rightarrow}
\newcommand{\PP}{\mathbb{P}}
\newcommand{\sse}{\subseteq}
\newcommand{\eps}{\epsilon}
\newcommand{\N}{\mathcal{N}}
\newcommand{\poly}{\textup{poly}}

\newcommand{\dom}{\textup{dom}}

\renewcommand{\thesubsection}{\thesection.\roman{subsection}}


% auto sized delimiters
\newcommand{\Br}[1]{\left\{#1\right\}}
\newcommand{\br}[1]{\left[#1\right]}
\newcommand{\pr}[1]{\left(#1\right)}
\newcommand{\ceil}[1]{\left\lceil#1\right\rceil}
\newcommand{\floor}[1]{\left\lfloor#1\right\rfloor}
\newcommand{\abs}[1]{\left|#1\right|}
\newcommand{\sgn}{\textup{sgn}}

%default delimiter for Pr and E
\DeclarePairedDelimiter{\defaultDelim}{[}{]}

\DeclareMathOperator{\capPr}{Pr}
\renewcommand{\Pr}[2][]{\capPr_{#1}\defaultDelim*{#2}}
\DeclareMathOperator{\capE}{E}
\newcommand{\E}[2][]{\capE_{#1}\defaultDelim*{#2}}
\DeclareMathOperator{\capVar}{Var}
\newcommand{\Var}[2][]{\capVar_{#1}\defaultDelim*{#2}}

%\DeclareMathOperator*{}{} puts subscript below


%%%%%%%%%%%%%%%%%%%%%%%%%%%%%%%%%%%%%%%%%%%%%%%%%
\begin{document}
% \problemset{\problemsetnum}{\duedate}{\studentname}
{\bf \noindent  Practice problems (don't turn in):}
\begin{enumerate}
	\item[1.] [DPV] Problem 7.10 (max-flow = min-cut example)
	
	\item[2.] [DPV] Problem 7.17  (bottleneck edges)  \\
	(Part (e) where you devise an algorithm to find bottleneck edges 
	is more challenging but a nice problem to try.)
	
	\item[3.] [DPV] Problem 7.19 (verifying max-flow)
	
	\item[4.] For a bipartite graph $G=(V_1\cup V_2,E)$ where $|V_1|=|V_2|=n$ a {\em 
	perfect matching} is a subset $S$ of edges where each vertex is incident exactly 1
	edge in $S$.  In other words, it's a matching of size $n$.   Given a bipartite graph $G$
	show how to determine if $G$ has a perfect matching by a reduction to the max-flow 
	problem.  So given $G$ define an input to the max-flow problem, and given a max-flow
	for this input how do you determine if the original graph $G$ has a perfect matching or not?
	What is the running time of your algorithm? \\
	(For hints see [DPV] Chapter 7.3 (Bipartite matching) and the beginning of Problem 7.24.)

	
\end{enumerate}
%
%
%\newpage
%
%
%\begin{problem} {[DPV] Problem 7.17 (e)}
%	Give an efficient algorithm to identify all bottleneck edges in a network. \textbf{Do as fast as possible}.\\
%	\textit{(Hint: Start by running the usual network flow algorithm, and then examine the residual graph.)}\\\\
%	\textbf{Answer:}
%
%\end{problem}

\newpage
\begin{problem} {[DPV] Problem 7.18 (a) (b)}
	Solve the following problems by reducing to the original max-flow problem.
	In other words, explain how to solve the
new flow variant problem using an algorithm for solving max-flow as a black-box.
	Explain how to take an input for the new problem and define an input for the original max-flow problem.  Then given a max-flow $f^*$ to this input you just defined, explain how to get the solution
	to the new problem.
	\begin{itemize}
		\item[(a)] There are many sources and many sinks, and we wish to maximize the total flow from all sources to all sinks.
		\item[(b)] Each vertex also has a capacity on the maximum flow that can enter it.
	\end{itemize}
	\textbf{Answer:}
\end{problem}

\newpage
\begin{problem} {[DPV] Problem 5.22 (a)}
	Prove the following property carefully:\\
	Pick any cycle in the graph (denote it by $C$), and let $e^*$ be the heaviest edge in that cycle $C$.  Thus, $w(e^*)\geq w(e')$ for all $e'\in C$. 
	Then there is a minimum spanning tree $T'$ that does not contain $e^*$.\\\\
	Hint: Take a MST $T$ which contains $e^*$.  Construct a new tree $T'$ which does not
	contain $e^*$ and $w(T')\leq w(T)$.   
	\\
	\textbf{Answer:}
	
	
\end{problem}

 \newpage
 \begin{problem}{[DPV] Problem 5.9 (d)}
 
 If the lightest edge in a graph is unique, then it must be part of every MST.

\end{problem}

Is the above statement True or False?  
\\
If True, then prove that is correct (explain why it always holds).  Or if it is
False then give a counterexample (show a graph where it does not hold).  \\
	\textbf{Answer:}


\end{document}