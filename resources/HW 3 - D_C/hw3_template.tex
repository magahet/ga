\documentclass{article}
\usepackage[T1]{fontenc}
\usepackage{amssymb, amsmath, graphicx, subfigure, enumerate}
\usepackage{amsthm,alltt} 
\usepackage[margin=1.25in]{geometry} %geometry (sets margin) and other useful packages
\usepackage{graphicx,ctable,booktabs}
\usepackage{mathtools}
\usepackage[boxed]{algorithm2e}
\usepackage{mathdots}
\usepackage{fancyhdr} %Fancy-header package to modify header/page numbering
\usepackage{cleveref}

\setlength{\oddsidemargin}{.25in}
\setlength{\evensidemargin}{.25in}
\setlength{\textwidth}{6in}
\setlength{\topmargin}{-0.4in}
\setlength{\textheight}{8.5in}



\newcommand{\heading}[6]{
  \renewcommand{\thepage}{\arabic{page}} % used to be #1-\arabic{page}
  \noindent
  \begin{center}
  \framebox{
    \vbox{
      \hbox to 5.78in { \textbf{#2} \hfill #3 }
      \vspace{4mm}
      \hbox to 5.78in { {\Large \hfill #6  \hfill} }
      \vspace{2mm}
      \hbox to 5.78in { \textit{Instructor: #4 \hfill #5} }
    }
  }
  \end{center}
  \vspace*{4mm}
}

%Redefining sections as problems
\makeatletter
\newenvironment{problem}{\@startsection
       {section}
       {2}
       {-.2em}
       {-3.5ex plus -1ex minus -.2ex}
       {2.3ex plus .2ex}
       {\pagebreak[3]%forces pagebreak when space is small; use \eject for better results
       \large\bf\noindent{Problem }
       }
       }
       %{%\vspace{1ex}\begin{center} \rule{0.3\linewidth}{.3pt}\end{center}}
       %\begin{center}\large\bf \ldots\ldots\ldots\end{center}}
\makeatother


\newtheorem{theorem}{Theorem}[section]
\newtheorem{definition}[theorem]{Definition}
\newtheorem{remark}[theorem]{Remark}
\newtheorem{lemma}[theorem]{Lemma}
\newtheorem{corollary}[theorem]{Corollary}
\newtheorem{proposition}[theorem]{Proposition}
\newtheorem{claim}[theorem]{Claim}
\newtheorem{observation}[theorem]{Observation}
\newtheorem{fact}[theorem]{Fact}
\newtheorem{assumption}[theorem]{Assumption}

%\newenvironment{proof}{\noindent{\bf Proof:} \hspace*{1mm}}{
% \hspace*{\fill} $\Box$ }
%\newenvironment{proof_of}[1]{\noindent {\bf Proof of #1:}
% \hspace*{1mm}}{\hspace*{\fill} $\Box$ }
%\newenvironment{proof_claim}{\begin{quotation} \noindent}{
% \hspace*{\fill} $\diamond$ \end{quotation}}

\newcommand{\problemset}[2]{\heading{#1}{\classname}{#2}{Problem Set #1}} % Don't change this line
%%%%%%%%%%%%%%%%%%%%%%%%%% Change this stuff below, don't change the line above this one
\newcommand{\problemsetnum}{3}            % problem set number
\newcommand{\duedate}{09/18/2017} % problem set deadline
\newcommand{\studentname}{Student Name: }      % name of student (i.e., you)
\newcommand{\classname}{CS 8803 GA -- HW 3.  \ \ \ Name:   }
%\newcommand{\instructor}{Prof. Eric Vigoda}
%%%%%%%%%%%%%%%%%%%%%%%%%%

\pagestyle{fancy}
%\addtolength{\headwidth}{\marginparsep} %these change header-rule width
%\addtolength{\headwidth}{\marginparwidth}
\lhead{\classname} %Problem \thesection}
\chead{} 
\rhead{\thepage} 
%\lfoot{\small\scshape \classname}
\cfoot{} 
%\rfoot{\footnotesize PS \#\problemsetnum} 
\renewcommand{\headrulewidth}{.3pt} 
\renewcommand{\footrulewidth}{.3pt}
\setlength\voffset{-0.25in}
\setlength\textheight{648pt}


\newcommand{\sit}{\shortintertext}
\newcommand\deq{\mathrel{\overset{\makebox[0pt]{\mbox{\normalfont\tiny\sffamily def}}}{=}}}
\newcommand{\ones}{\mathbbm{1}}
\newcommand{\e}{\vec{e}}
\newcommand{\tr}{\text{tr}}
\newcommand{\bs}{\boldsymbol}
\mathchardef\mhyphen="2D
\newcommand{\C}{\mathbb{C}}
\newcommand{\R}{\mathbb{R}}
\newcommand{\II}{\mathcal{I}}
\newcommand{\FF}{\mathcal{F}}
\newcommand{\X}{\mathcal{X}}
\newcommand{\Y}{\mathcal{Y}}
\newcommand{\ra}{\rightarrow}
\newcommand{\Ra}{\Rightarrow}
\newcommand{\PP}{\mathbb{P}}
\newcommand{\sse}{\subseteq}
\newcommand{\eps}{\epsilon}
\newcommand{\N}{\mathcal{N}}
\newcommand{\poly}{\textup{poly}}

\newcommand{\dom}{\textup{dom}}

\renewcommand{\thesubsection}{\thesection.\roman{subsection}}


% auto sized delimiters
\newcommand{\Br}[1]{\left\{#1\right\}}
\newcommand{\br}[1]{\left[#1\right]}
\newcommand{\pr}[1]{\left(#1\right)}
\newcommand{\ceil}[1]{\left\lceil#1\right\rceil}
\newcommand{\floor}[1]{\left\lfloor#1\right\rfloor}
\newcommand{\abs}[1]{\left|#1\right|}
\newcommand{\sgn}{\textup{sgn}}

%default delimiter for Pr and E
\DeclarePairedDelimiter{\defaultDelim}{[}{]}

\DeclareMathOperator{\capPr}{Pr}
\renewcommand{\Pr}[2][]{\capPr_{#1}\defaultDelim*{#2}}
\DeclareMathOperator{\capE}{E}
\newcommand{\E}[2][]{\capE_{#1}\defaultDelim*{#2}}
\DeclareMathOperator{\capVar}{Var}
\newcommand{\Var}[2][]{\capVar_{#1}\defaultDelim*{#2}}

%\DeclareMathOperator*{}{} puts subscript below


%%%%%%%%%%%%%%%%%%%%%%%%%%%%%%%%%%%%%%%%%%%%%%%%%
\begin{document}
%\problemset{\problemsetnum}{\duedate}{\studentname}

{\bf \noindent  Practice problems (don't turn in):}
\begin{enumerate}
\item[1.] [DPV] Problem 2.8 (FFT practice),
\item[2.] [DPV] Problem 2.17 (fixed point).
\end{enumerate}

\bigskip


\begin{problem} {[DPV] Problem 2.9 part (b)}
	Do polynomial multiplication by FFT for the pair of polynomials $1 + x + 2x^2$ and $2 + 3x$. \\
	(You might try part (a) for practice.) \\ \\ 
\textbf{Answer:  (Yes, show your work!)}

\end{problem}

\newpage
\begin{problem} {[DPV] Problem 2.16  (find x in an infinite array)}
	\textbf{Answer:  (Explain your algorithm in words and analyze its running time)}
	
\end{problem}

\newpage
\begin{problem} {}
	You are given an array $A= [1, \dots, n]$ of $n$ positive numbers that represent the price of a particular stock on $n$ days. You want to buy the stock at one day and sell on a later day. Your goal is to determine what would have been the best pair of days for buying/selling. You simply have to output the difference in price.  You can assume $n$ is a power of 2.
	Here is an example: $A = [10, 15, 6, 3, 7, 12, 2, 9]$. Then the optimal decision would be to purchase on day 4 for \$3 and sell on day 6 for \$12, so your algorithm should output \$9.  \\
	\begin{enumerate}
		\item [] \textbf{3. Part A:}\\\\
		Suppose you want to buy in the first $n/2$ days and you want to sell in the last $n/2$ days. Give an $O(n)$ time algorithm for finding the best pair of days for buying/selling under this restriction. Explain your algorithm in words and justify/explain its running time.\\\\
		\textbf{Answer:}
		
		
		\newpage
		\item[] \textbf{3. Part B:}\\\\
		Give a divide and conquer algorithm with running time $O(n\log n)$ for finding the best pair of days for buying/selling. (There are no restrictions on the days, except that you need to buy at an earlier date than you sell.) Explain your algorithm in words and analyze your algorithm, including stating and solving the relevant recurrence.\\
		\textit{Hint: Your algorithm should use your solution to part a.}\\\\
		\textbf{Answer:}
		
		\newpage
		\item[] \textbf{3. Extra Credit:}\\\\
		Give a divide and conquer algorithm with running time $O(n)$ for finding the best pair of days for buying/selling.\\
		\textit{Hint: Modify the problem to obtain additional info from the subproblems so that you can do the "merge" part in $O(1)$ time.}\\\\
		\textbf{Answer:}
		
	\end{enumerate}
	
\end{problem}

\newpage
\begin{problem} {Integer multiplication using FFT}
	\begin{enumerate}
		\item [] \textbf{4. Part A:}\\\\
		Given an $n$-bit integer number $a$ where $a = a_0a_1\dots a_{n-1}$, define a polynomial $A(x)$ where $A(2) = a$.\\\\
		\textbf{Answer:}
		
		
		\newpage
		\item[] \textbf{4. Part B:}\\\\
		Given 2 $n$-bit integers $a$ and $b$, give an algorithm to multiply them in $O(n\log{n})$ time.  Use the FFT algorithm from class as a black-box (i.e. don't rewrite the code, just say run FFT on ...).  Explain your algorithm in words and its running time.\\\\
		\textbf{Answer:}
	\end{enumerate}
	
\end{problem}

\newpage
\begin{problem} {Deterministic Algorithm for finding the Median}
	For the deterministic $O(n)$ time algorithm for finding the median presented in class on Thursday, February 2, suppose we broke the array into groups of size 3 or 7 (instead of 5).\\
	Does groups of 3 or 7 work?  Why? Make sure to state the recurrence $T(n)$ for the modified algorithm for each case and explain why it does or does not solve to $O(n)$.\\\\
	\textbf{Answer:}
	
\end{problem}


\end{document}