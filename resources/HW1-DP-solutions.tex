\documentclass[12pt]{article}

\usepackage{amsmath}    % need for subequations
\usepackage{graphicx}   % need for figures
\usepackage{verbatim}   % useful for program listings
\usepackage{subfigure}  % use for side-by-side figures
\usepackage{hyperref}   % use for hypertext links, including those to external documents and URLs
\usepackage{clrscode}
\usepackage{algorithmic}
\usepackage{amssymb}
\usepackage{algorithm}
\usepackage{url,hyperref}
\usepackage{amsthm}

% don't need the following. simply use defaults
%\setlength{\baselineskip}{50.0pt}    % 16 pt usual spacing between lines
%\setlength{\parskip}{3pt plus 2pt}
%\setlength{\parindent}{20pt}
%\setlength{\oddsidemargin}{0.5cm}
%\setlength{\evensidemargin}{0.5cm}
%\setlength{\marginparsep}{0.75cm}
%\setlength{\marginparwidth}{2.5cm}
%\setlength{\marginparpush}{1.0cm}
%\setlength{\textwidth}{150mm}
\newtheorem{theorem}{Theorem}[section]
\newtheorem{lemma}[theorem]{Lemma}

\algsetup{indent=1cm}

\begin{document}

\begin{center}
{\large 
Solutions for HW 1 Dynamic Programming Problems} \\
\end{center}

{\noindent \bf [DPV] Problem 6.1 --  Max contiguous subsequence}

{\noindent \bf Solution:}
Let's first solve the problem of finding the maximum sum obtainable by a contiguous
subsequence, and then we'll see it's easy to modify that algorithm to output a valid
subsequence.

For solving the problems using dynamic programming, we first identify the subproblems and then
we try to express a recurrence for the solution to the subproblem in
terms of smaller subproblems.
Our first attempt is often to consider {\em prefixes} for the subproblem.
For the first $i$ numbers, find the maximum contiguous subsequence within it.
However, one will realize that the recurrence is hard to write down because we have to
maintain  the property that the solution is contiguous.
As in the longest increasing
subsequence problem, we have to strengthen the subproblem so that we consider
the best solution that includes the number $a_i$.

We define the subproblem as:
\begin{eqnarray*}
S(i) = & \mbox{ the maximum sum attainable by a contiguous subsequence from the list } \\
& \mbox{ of numbers $a_1,a_2,a_3,...,a_i$ with the extra constraint that } \\
& \mbox{ the contiguous subsequence has to end with $a_i$. }
\end{eqnarray*}

The reason to force the subproblem to have this extra condition is that, just
as for the longest increasing subsequence problem that we saw in class, it is 
necessary in order to write a recursive formula for $S(i)$ in terms of $S(1),\dots, S(i-1)$.

Let's look at the example in the book, $A = [5, 15, -30, 10, -5, 40, 10]$.
Then, $S(1) = 5$ since $5$ itself is the only option.
$S(2)=20$ since $5,15$ has larger sum than $15$ by itself.
Similarly, $S(3) = -10$ since $5+15-30 =-10$ is larger than $-30$ by itself.
Note, $S(4)=10$ since that is larger than $5+15-30+10=0$.
The whole table is $S = [5, 20, -10, 10, 5, 45, 55]$.

Notice that to compute $S(i)$ we either consider adding $a_i$ to the optimal
subsequence ending at $a_{i-1}$ or we consider $a_i$ by itself.
Hence, the recurrence is the following:
\[ S(i) = \max\{ a_i, a_i+S(i-1)\}.
\]

It is easy to see that once we define these subproblems, the final solution to the original problem is the best solution among $S(1), S(2),..., S(n)$ which can be computed in $O(n)$ time assuming that all the subproblems are solved.
The pseudocode for filling the table is below.

To obtain a valid subsequence, as in the longest increasing subsequence problem
that we did in class, we just have to ``backtrack'' in the table $S$ to determine if the
optimal subsequence only includes $a_i$, or if it continues to the optimal subsequence ending at $a_{i-1}$.
To this end we add use the table $prev()$ to keep track of which value achieves
the maximum in the recurrence.

\begin{algorithm}[h!]
\caption{Max sum of contiguous subsequence($a_1,a_2,\dots,a_n$)}
\begin{algorithmic}
\STATE $S(0) = 0$.
\FOR{$i=1$ to $n$}
\IF{$S(i-1) + a_i \ge a_i$}
\STATE $S(i) = S(i-1)+a_i$
\STATE $prev(i) = i-1$
\ELSE
\STATE $S(i) = a_i$
\STATE $prev(i) = NULL$
\ENDIF
\ENDFOR
\STATE $max=1$
\FOR{$i=2$ to $n$}
\IF{$S(i)>S(max)$} 
\STATE $max=i$
\ENDIF
\ENDFOR
\RETURN $S(max)$
\end{algorithmic}
\end{algorithm}


The running time is $O(n)$.

To output a contiguous subsequence obtaining the maximum sum we add
the following to the algorithm.

\begin{algorithm}[h!]
\caption{Outputting a contiguous subsequence of maximum sum}
\begin{algorithmic}
\STATE $i=max$
\STATE $output(i)$
\WHILE{$prev(i)\neq NULL$}  
\STATE $i = prev(i)$
\STATE  $output(i)$
  \ENDWHILE
\end{algorithmic}
\end{algorithm}






%\vspace{1in}
\newpage

{\noindent \bf [DPV] Problem 6.2 --  Hotel stops}

 {\noindent \bf Solution:}

As in Problem 1, we strengthen the subproblem so that we consider
 the best subsequence which includes hotel $i$.

 Therefore, we define the following subproblem.  

 For $1\le i \le n$, let
\begin{eqnarray*}
  P(i)  = & \mbox{ minimum penalty obtainable for the trip} \\
&   \mbox{ from mile 0
 to mile $a_i$ with the last stop at hotel $i$.}
 \end{eqnarray*}

 To figure out the recurrence, we consider the choices for the last
 hotel stop prior to $i$.  Say that this penultimate stop is at hotel $k$ which is at mile $a_k$.  
 Then the cost for the last segment of the trip is $(200 - (a_i - a_k))^2$,
 and it's $P(k)$ for the beginning segment of the trip ending at hotel $k$.
 Therefore, the total cost for this trip with a penultimate stop at the $k$-th hotel is
 $P(k) + (200 - (a_i - a_k))^2$.
We then try all possibilities for $k$, so all $k$ where $0\le k\le i-1$.
Note, $k=0$ corresponds to having no stops prior to the $i$-th hotel.
Finally, the recurrence is
\[ P(i) = \min_{k} \{P(k) + (200 - (a_i - a_k))^2 : 0\le k\le i-1\}
\]
This says that $P(i)$ is the minimum over $k$, where $k$ is
restricted to be at least $0$ and at most $i-1$, and we're
minimizing the quantity $P(k) +  (200 - (a_i - a_k))^2$.
The base case is $P(0) = 0$.

Below is the pseudocode for filling the table.  To implement the minimum over $k$,
we first set $P(i)$ to the value from the $k=0$ case,
then we try other choices of $k$ and redefine $P(i)$ if a smaller
cost is found.

To find a set of locations with the minimum penalty we add the $prev(i)$ to
store the $k$ which achieves the minimum in the recurrence, and then we backtrack.

\begin{algorithm}[h!]
\caption{Hotel stops($a_1,a_2,\dots,a_n$)}
\begin{algorithmic}
\STATE $P(0) = 0$
\FOR{$i=1$ to $n$}
\STATE $P(i) = (200 - a_i)^2$
\STATE $prev(i) = NULL$
\FOR{$k=1$ to $i-1$}
\IF{$P(i) > P(k) + (200 - (a_i - a_k))^2$}
\STATE $P(i)=P(k) + (200 - (a_i - a_k))^2$
\STATE $prev(i) = k$
\ENDIF
\ENDFOR
\ENDFOR

\COMMENT{**$P(n)$ is the minimum penalty obtainable**}

\COMMENT{**The following outputs a set of locations obtaining the minimum penalty**}
\STATE $i=n$
\STATE $output(i)$
\WHILE{$prev(i)\neq NULL$}  
\STATE $i = prev(i)$
\STATE  $output(i)$
  \ENDWHILE
\end{algorithmic}
\end{algorithm}

The running time is $O(n^2)$.


\newpage

{\noindent \bf [DPV] Problem 6.3 -- YuckDonald's}

{\noindent \bf Solution:}
Again this problem is exactly the same flavor as Problems 1 and~2.
We define the subproblem as:
\begin{eqnarray*}
  L(i) = & \mbox{maximum profit from a valid subset of locations $m_1,m_2,...,m_i$} \\
& \mbox{ with the extra
constraint that $m_i$ has to be included.}
\end{eqnarray*}
 The final solution to the problem is therefore
$\max_i L(i)$.

The base case is $L(0) = 0$. To define the recurrence for $L(i)$, 
since we are putting a restaurant at location $m_i$, we gain profit $p_i$ from it,
and then the penultimate location must be at least $k$ miles away.
Hence, the penultimate location can only be those $j$ where $m_j\leq m_i - k$.
We try all possibilities for this penultimate location $m_j$, and
the maximum profit we obtain for a subset of locations $1,\dots,j$ is
captured in $L(j)$.  Hence, if the penultimate location is $m_j$ the total
profit we obtain is $p_i+L(j)$.  Therefore, the recurrence is the following:
\[
L(i) = p_i + \max_{j}  \{L(j) : m_j \le m_i - k \}.
\]

Here is the pseudocode for filling the table:
\begin{algorithm}[h!]
\caption{YuckDonalds($a_1,a_2,\dots,a_n$)}
\begin{algorithmic}
\STATE $L(0) = 0$.
\FOR{$i=1$ to $n$}
\STATE $L(i) = p_i$.
\FOR{$j=1$ to $i-1$}
\IF {$m_j \le m_i - k$}
\IF{$L(i) < L(j) + p_i$}
\STATE $L(i)=L(j) + p_i$
\ENDIF
\ENDIF
\ENDFOR
\ENDFOR
\RETURN $\max_i L(i)$
\end{algorithmic}
\end{algorithm}

The running time is $O(n^2)$.

\medskip

{\bf Alternative Solution:}

One can also solve the problem by defining the subproblem as follows:
\[
  P(i) =  \mbox{maximum profit from a valid subset of locations }  m_1,m_2,...,m_i
  \]
Hence, we have dropped the extra constraint that $m_i$ has to be included.
 In this case, the final solution to the problem is therefore simply $P(n)$.

To figure out the recurrence, we have two cases, either $m_i$ is included or 
it is not included.  If it is not included, then the best subset of locations $m_1,\dots,m_i$
is the same as the best subset of locations $m_1,\dots,m_{i-1}$.  Hence, in this
case, $P(i)=P(i-1)$.  If location $m_i$ is included then the penultimate location
must be at least $k$ miles away.  Let $last(i)$ denote the last possible location 
that is at least $k$ miles from $m_i$, in other words:
\[  last(i) = \max\{\ell: m_\ell\leq m_i-k\}.
\]  Now if location $m_i$ is included, then
the remaining solution must be a subset of locations $m_1,\dots,m_\ell$
where $\ell=last(i)$.  Hence, in this case where location $m_i$ is included, we have
that $P(i) = p_i + P(last(i))$.
Therefore, the recurrence is the following:
\[
P(i) = \max\{ P(i-1), p_i + P(last(i)) \}
\]

It is easy to modify the earlier pseudocode to obtain an $O(n^2)$ time
solution.  In fact, by first calculating $last(i)$ for $i=1\rightarrow n$, 
and then calculating $P(i)$ for $i=1\rightarrow n$, one obtains an $O(n)$
time solution.

Here is the pseudocode for the faster solution:
\begin{algorithm}[h!]
\caption{FasterYuckDonalds($a_1,a_2,\dots,a_n$)}
\begin{algorithmic}
\STATE $a_0 = 0$.
\STATE $\ell = 0$.
\FOR{$i=1$ to $n$}
\WHILE{$\ell<i-1$ and $m_{\ell+1}\leq m_i-k$}
\STATE $\ell = \ell+1$.
\ENDWHILE
\STATE $last(i) = \ell$.
\ENDFOR
\STATE $P(0) = 0$.
\FOR{$i=1$ to $n$}
\STATE $P(i) = \max\{ P(i-1), p_i + P(last(i))\}$.
\ENDFOR
\RETURN $P(n)$
\end{algorithmic}
\end{algorithm}


\newpage

{\noindent \bf [DPV] Problem 6.17 --  Making change I}


{\noindent \bf Solution:}

In this problem,
 you have $n$ denominations
$x_1,\dots,x_n$ (with unlimited supply of each)
and a value $V$, and you are asked to determine in $O(nV)$ time whether there
is a set of coins with total value $V$.
This problem is similar to the knapsack problem (with repetition).

We make a one dimensional table.  For $0\le w\le V$, let
\[  S(w) = \{\mbox{{\tt TRUE} or {\tt FALSE} whether there is a subset of coins
with total value $w$}\}.
\]
By considering the last coin used, we get that $S(w)$ is {\tt TRUE} if
there is a denomination $1\le i \le n$ where $S(w-x_i)$ is {\tt TRUE}.
Hence, we get the following recurrence, where $\bigvee$ denotes OR.
For $0\le w\le V$,
\[ S(w) = \bigvee_{i}\{ S(w-x_i) :
1\le i\le n, x_i\le w\}
\]
This yields an $O(nV)$ time algorithm with 2 for-loops.

\begin{algorithm}[h!]
\caption{Coin Changing I} 
\begin{algorithmic}
\STATE $S(0) = {\tt TRUE}$.
\FOR{$j=1$ to $v$}
\STATE $S(j) = {\tt FALSE}$.
\ENDFOR
\FOR{$i=1$ to $v$}
\FOR{$j=1$ to $n$}
\IF {$i - x_j \geq 0$}
\STATE $S(i) \leftarrow S(i-x_j) \vee S(i)$
\ENDIF
\ENDFOR
\ENDFOR
\RETURN $S(v)$
\end{algorithmic}
\end{algorithm}

\newpage

{\noindent \bf [DPV] Problem 6.18 -- Making change II}

{\noindent \bf Solution:}
This problem is very similar to the knapsack problem without
repetition that we saw in class.

First of all, let's identify the subproblems.
Since each denomination is used at most once, consider the situation
for $x_n$.
There are two cases, either
\begin{itemize}
\item We do not use $x_n$ then we need to use
a subset of $x_1,\dots,x_{n-1}$ to form value~$v$;
\item We use $x_n$ then we need to use
a subset of $x_1,\dots,x_{n-1}$ to form value $v-x_n$.
Note this case is only possible if $x_n \leq v$.
\end{itemize}
If either of the two cases is {\tt TRUE}, then the answer for the original problem is {\tt TRUE}, otherwise it is {\tt FALSE}. These two subproblems can depend further on
some subproblems defined in the same way recursively, namely,
 a subproblem considers a prefix of the denominations and some value.

We define a $n\times v$ sized table $D$ defined as:
\begin{eqnarray*}
D(i,j) = & \{\mbox{{\tt TRUE} or {\tt FALSE} where 
there is a subset of the
coins of } \\
& \mbox{ denominations $x_1$,...,$x_i$ to form the value $j$.} \}
\end{eqnarray*}
Our final answer is stored in the entry $D(n,v)$.

Analogous to the above scenario with denomiation $x_n$ we have the
following recurrence relation for $D(i,j)$.
For $i>0$ and $j>0$ then we have:
\[
D(i,j) =
\begin{cases}
D(i-1,j) \vee D(i-1,j-x_i) & \mbox{if } x_i\le j \\
D(i-1,j) & \mbox{if } x_i>j.
\end{cases}
\]
 (Recall, $\vee$ denotes Boolean OR.)

The base cases are $D(0,0)={\tt TRUE}$ and
for all $j=1,2,\dots,v$,
$D(0,j) = {\tt FALSE}$.

The algorithm for filling in the table is the following.
\begin{algorithm}[h!]
\caption{Coin Changing II}
\begin{algorithmic}
\STATE $D(0,0) = {\tt TRUE}$.
\FOR{$j=1$ to $v$}
\STATE $(0,j) = {\tt FALSE}$.
\ENDFOR
\FOR{$i=1$ to $n$}
\FOR{$j=0$ to $v$}
\IF{$x_i\le j$}
\STATE $D(i,j) \leftarrow D(i-1,j) \vee D(i-1,j-v_i)$
\ELSE \STATE
$D(i,j)  \leftarrow D(i-1,j)$
\ENDIF
\ENDFOR
\ENDFOR
\RETURN $D(n,v)$
\end{algorithmic}
\end{algorithm}

Each entry takes $O(1)$ time to compute, and there are
$O(nv)$ entries.  Hence, the total running time is $O(nv)$.




\end{document}
