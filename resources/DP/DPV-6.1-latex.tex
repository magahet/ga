\documentclass[12pt]{article}

\usepackage{amsmath}    % need for subequations
\usepackage{graphicx}   % need for figures
\usepackage{verbatim}   % useful for program listings
\usepackage{subfigure}  % use for side-by-side figures
\usepackage{hyperref}   % use for hypertext links, including those to external documents and URLs
\usepackage{clrscode}
\usepackage{algorithmic}
\usepackage{amssymb}
\usepackage{algorithm}
\usepackage{url,hyperref}
\usepackage{amsthm}

% don't need the following. simply use defaults
%\setlength{\baselineskip}{50.0pt}    % 16 pt usual spacing between lines
%\setlength{\parskip}{3pt plus 2pt}
%\setlength{\parindent}{20pt}
%\setlength{\oddsidemargin}{0.5cm}
%\setlength{\evensidemargin}{0.5cm}
%\setlength{\marginparsep}{0.75cm}
%\setlength{\marginparwidth}{2.5cm}
%\setlength{\marginparpush}{1.0cm}
%\setlength{\textwidth}{150mm}
\newtheorem{theorem}{Theorem}[section]
\newtheorem{lemma}[theorem]{Lemma}

\algsetup{indent=1cm}

\begin{document}

\begin{center}
{\large 
Solutions for [DPV] Practice Dynamic Programming Problems} \\
\end{center}

{\noindent \bf [DPV] Problem 6.1 --  Max contiguous subsequence}

{\noindent \bf Solution:}
Let's first solve the problem of finding the maximum sum obtainable by a contiguous
subsequence, and then we'll see it's easy to modify that algorithm to output a valid
subsequence.

For solving the problems using dynamic programming, we first identify the subproblems and then
we try to express a recurrence for the solution to the subproblem in
terms of smaller subproblems.
Our first attempt is often to consider {\em prefixes} for the subproblem.
For the first $i$ numbers, find the maximum contiguous subsequence within it.
However, one will realize that the recurrence is hard to write down because we have to
maintain  the property that the solution is contiguous.
As in the longest increasing
subsequence problem, we have to strengthen the subproblem so that we consider
the best solution that includes the number $a_i$.

We define the subproblem as:
\begin{eqnarray*}
S(i) = & \mbox{ the maximum sum attainable by a contiguous subsequence from the list } \\
& \mbox{ of numbers $a_1,a_2,a_3,...,a_i$ with the extra constraint that } \\
& \mbox{ the contiguous subsequence has to end with $a_i$. }
\end{eqnarray*}

The reason to force the subproblem to have this extra condition is that, just
as for the longest increasing subsequence problem that we saw in class, it is 
necessary in order to write a recursive formula for $S(i)$ in terms of $S(1),\dots, S(i-1)$.

Let's look at the example in the book, $A = [5, 15, -30, 10, -5, 40, 10]$.
Then, $S(1) = 5$ since $5$ itself is the only option.
$S(2)=20$ since $5,15$ has larger sum than $15$ by itself.
Similarly, $S(3) = -10$ since $5+15-30 =-10$ is larger than $-30$ by itself.
Note, $S(4)=10$ since that is larger than $5+15-30+10=0$.
The whole table is $S = [5, 20, -10, 10, 5, 45, 55]$.

Notice that to compute $S(i)$ we either consider adding $a_i$ to the optimal
subsequence ending at $a_{i-1}$ or we consider $a_i$ by itself.
Hence, the recurrence is the following:
\[ S(i) = \max\{ a_i, a_i+S(i-1)\}.
\]

It is easy to see that once we define these subproblems, the final solution to the original problem is the best solution among $S(1), S(2),..., S(n)$ which can be computed in $O(n)$ time assuming that all the subproblems are solved.
The pseudocode for filling the table is below.

To obtain a valid subsequence, as in the longest increasing subsequence problem
that we did in class, we just have to ``backtrack'' in the table $S$ to determine if the
optimal subsequence only includes $a_i$, or if it continues to the optimal subsequence ending at $a_{i-1}$.
To this end we add use the table $prev()$ to keep track of which value achieves
the maximum in the recurrence.

\begin{algorithm}[h!]
\caption{Max sum of contiguous subsequence($a_1,a_2,\dots,a_n$)}
\begin{algorithmic}
\STATE $S(0) = 0$.
\FOR{$i=1$ to $n$}
\IF{$S(i-1) + a_i \ge a_i$}
\STATE $S(i) = S(i-1)+a_i$
\STATE $prev(i) = i-1$
\ELSE
\STATE $S(i) = a_i$
\STATE $prev(i) = NULL$
\ENDIF
\ENDFOR
\STATE $max=1$
\FOR{$i=2$ to $n$}
\IF{$S(i)>S(max)$} 
\STATE $max=i$
\ENDIF
\ENDFOR
\RETURN $S(max)$
\end{algorithmic}
\end{algorithm}


The running time is $O(n)$.

To output a contiguous subsequence obtaining the maximum sum we add
the following to the algorithm.

\begin{algorithm}[h!]
\caption{Outputting a contiguous subsequence of maximum sum}
\begin{algorithmic}
\STATE $i=max$
\STATE $output(i)$
\WHILE{$prev(i)\neq NULL$}  
\STATE $i = prev(i)$
\STATE  $output(i)$
  \ENDWHILE
\end{algorithmic}
\end{algorithm}






\end{document}
