 \documentclass[12pt,twoside]{article}

\usepackage{algorithm,fancyhdr,alltt,lastpage}
\usepackage[noend]{algorithmic}
%An alternative package for pseudcode is the following:
%\usepackage{pseudocode}    

%\usepackage{epsfig}
%\usepackage{amssymb}
%\usepackage{amsfonts}

\newcommand{\problem}[1]{\vspace{.1in} \noindent {\bf #1}.\\ }

\newcommand{\A}{\mathcal{A}}
\def\Prob#1{{\mathbf{Pr}\left({#1}\right)}}
\def\blue{\color{blue}}
\newcommand{\eps}{\epsilon}
\newcommand{\dist}{\mathrm{dist}}


\pagestyle{fancy}

%\fancyhead[LO]{Name:  }
%\fancyhead[CO]{ ID:  }
%\fancyhead[RO,RE]{Page \thepage \ of \pageref{LastPage}}
%\lfoot{}
%\cfoot{}
%\rfoot{}

\lhead{Name:  }
\chead{ ID:  }
\rhead{Page \thepage \ of \pageref{LastPage}}



\begin{document}


\newpage

\vspace*{1in}

{\bf (1a) } Define the entries of your table in words.  E.g., $T(i)$ is ..., or $T(i,j)$ is~....

\vspace{2in}

{\bf (1b) } State the recurrence for the entries of your table in
terms of smaller subproblems.

\newpage

{\bf (1c) } Write pseudocode for your algorithm to solve this problem.

\vspace{6in}

{\bf (1d) } Analyze the running time of your algorithm.




\end{document}
